% 2005-03-07
\documentclass[12pt]{article}
\pagestyle{empty}
\usepackage{geometry}
\geometry{letterpaper,tmargin=1in,bmargin=1in,lmargin=1in,rmargin=1in,headheight=0in,headsep=0in,footskip=.3in}

\newcommand{\Sec}[1]{Sec. \ref{#1}}

\begin{document}

\begin{center}
  \textbf{\Large The Rules of Uncle Baiting}
\end{center}

Although the game's called Uncle Baiting, it's perfectly permissible to bait Aunts, Nieces, Nephews,
Cousins, Sons, Daughters, and Friends --- even Grandparents;  this is an equal-opportunity game.

\section{Setting Up}

\begin{itemize}
\item At least two decks should be used, shuffled together.  If you're not playing the Jokers rule they should
  be removed.
  
\item At the beginning each player is dealt 7 cards.
  
\item The remaining cards are placed face down in the centre of the table forming the `pick-up pile'.
  When everyone is ready the dealer turns over the 
  top card and places it next to the pickup-pile forming a discard pile;%
\footnote{N.b. this card is `dead'; none of the special meanings of cards apply (e.g.
no-one has to pick up 4 cards if it's an ace; \Sec{specialCards}); if you 
`butt-in' (\Sec{buttingIn}) the butted-in card is also dead.}
the game has begun.

\end{itemize}

\section{How to Play}

The aim of the game is to get rid of all your cards.  On each turn you usually attempt to play one card face up on to
the top of the discard pile; it must either be the same suit (e.g. hearts; spades) or number (e.g.  3; Queen) as the
last card played.  The only exception to the follow-suit-or-number rule is a Jack --- see \Sec{specialCards}.  If you
can't play (or don't wish to play) you must take a card from the pick-up pile.

Initially play passes to the left; the player to the dealer's left starts.
Players should be given time to sort their hands before the game commences; no
penalties may be extracted for being slow at this stage.

When you have only one card left\footnote{I.e. when you play the penultimate card} you \emph{must} say ``Last
Card''; failure to do so results in a penalty (see \Sec{penalties}).  This penalty is picked up on your \textit{next} turn, the one when you would have won. Exception: if you are required to pick up on this turn, you collect your penalty and then play as usual; either pick up or play a 2.

When the pick-up pile is exhausted, turn over the discard pile (leaving the top card behind) and
place face-down as a new pick-up pile.  No penalties may be extracted while this is being
done.

\section{The Meaning of Cards}
\label{specialCards}

Some (well, most) cards are special.  If the card that's just been played is special you may be required to
do something other than simply play yourself.  By agreement before the game, you may decide to apply only
some of these rules (``Let's play aces, 2s, red and black 7s, 8s, 9s, Jacks, Kings, and Jokers'').

In the following examples we assume that the players are $A$, $B$, $C$, $D$, $E$, and $F$ in order of play,
and $6_A$ means that $A$ plays a 6.  If you're in `Queens mode' the plays are indicated as e.g. $X_{A1} X_{A2} X_{B1}$,
where $A1$ means $A$'s leading hand, in other words the hand that plays first.

We use the term `hand' to refer to a turn --- usually a hand is the same as a player, but not when
you're in Queens mode.

\begin{description}
\newcommand{\card}[1]{\hbox to 15mm{#1\hfil}}
\item[\card{Ace}] The following hand must pick up four cards.  The player of the ace must say ``Sorry'', although
  sincerity is neither required nor recommended.  If a player goes out with an ace they must still say ``Sorry''
  otherwise they will be required to pick up a penalty and the game continues.

  \item[\card{2}]
    The following hand must pick up two cards, or play another 2 in which case the next hand must
    pick up four cards (or play a third 2 --- the number of cards to pick up increases only arithmetically).

  \item[\card{6}]
    The direction of play reverses immediately, just as if you'd played a King.  Then, after two
    more hands have been played it reverses again.  In the simple case this means that play just goes backwards
    for two turns; If $C$ plays
    a 6, the result would be $6_C X_B X_A X_B X_C X_D X_E$.  Now consider what happens if a
    King's played;  you get $6_C K_B X_C X_B X_A$ or $6_C X_B K_A X_F X_E$.  With an 8,
    $6_C 8_B X_F X_A X_B$.

    If a second 6 is played while a 6 is active, the first 6's pending reversal is cancelled; e.g.  $6_C X_B
    6_A X_B X_C X_B X_A X_F$.  Note that in this case the net effect is that the order of play has reversed.

    N.b. 6s apply to hands, not players.  In Queens mode, you might have $6_{C2} X_{C1} X_{B2} X_{C1} X_{C2}$.

  \item[\card{7}] The next card played must be 5 or less if the 7 is red; Jack or higher if the 7's black. You
    must still follow suit, and, of course, you may butt-in on the original 7.
    
    N.b. Aces are low and may therefore be played following a red 7; 7s and Jacks are not 5 or less.%
    \footnote{The reason for the apparently asymmetrical ranges of permitted cards (Ace ... 5 v. Jack ... King)
    is that in fact there are four possible Jacks that can be played, so the ranges really are almost
    equal --- five cards against six.}

  \item[\card{8}]
    The next hand is skipped, e.g. $8_C X_E X_F$.

  \item[\card{9}]
    The play shifts to the previous player and continues in the original direction: $9_C X_B X_C X_D$.

    N.b. 9s apply to players, not hands: in Queens mode, $9_{C1} X_{B2} X_{C1} X_{C2} X_{D1}$.

  \item[\card{10}]
    Every other hand is skipped until another 10 is played: $10_C X_E X_A X_C$.  If there are an even number of
    players, this leads to only half the players getting a chance to play until another
    10 is played (or a Joker or 9; e.g.  $10_C X_E X_A 9_C X_B X_D X_F$).

    When going into 10s mode (i.e. starting to skip) the player must say, ``In''; when playing a
    10 that leaves 10s mode the player must say, ``Out''.  Failure to do so results, of course,
    in picking up a penalty card.  A player whose last card is a 10 must still say ``In'' or ``Out''
    before the game is over.

  \item[\card{Jack}]
    A Jack may be played even if it doesn't follow suit or number.  There are three situations
    when a Jack may \emph{not} be played:  after an Ace or 2; after a red 7; and as a player's last card.

    You \emph{must} announce the suit that the Jack represents (it may be the same as the
    current suit, or that of the Jack, if you so desire);  in other words after a Jack of
    Hearts, if the player announces ``Spades'' you must follow with a spade or another Jack.

    If you fail to announce the new suit the next player may play a card of any suit other than the Jack's own suit.%
    \footnote{The rationale for requiring a change of suit is to minimise the chance that
      a careless player will make a correct play by mistake.}
    After the penalty for not naming a suit has been duly awarded, play continues from the card on the top of the
    discard pile --- the owner of the Jack has missed their chance to choose a new suit.

  \item[\card{Queen}]
    In Queens mode, each player plays twice, once for their `leading' and once for their
    `trailing' hand --- the leading hand plays first.

    When going into Queens mode (i.e. starting to play two hands per player) the player must say, ``In''; when playing
    the Queen that leaves Queens mode the player must say, ``Out'', even if the Queen is the the player's
    last card. Failure to do so is rewarded by picking up a penalty card.  

    Note that the combination of 10s and Queens mode feels normal at first sight as each player plays a single hand in
    turn.  However, when another 10 is played you need to know whether you were playing on leading or trailing
    hands: $X_{A1} X_{B1} X_{C1} 10_{D1} X_{D2} X_{E1} X_{E2}$ but $X_{A2} X_{B2} X_{C2} 10_{D2} X_{E1} X_{E2}$.

  \item[\card{King}]
    The direction of play is reversed: $X_C K_D X_C X_B$.  In Queens mode, this can result in playing
    three hands in succession: $X_{C2} X_{B1} K_{B2} X_{B1} X_{C2}$.

  \item[\card{Joker}]

    Playing a Joker doesn't count as a turn, rather it may be played on any other card to modify its
    behaviour; specifically it makes the card behave as if the following player had played it. That is, if A
    plays a card X and a Joker is added, play continues as if B had played the X.  You may play a Joker even
    when it isn't your turn; the only exceptions are that you may not go out with a Joker and you may not play
    a Joker to avoid picking up cards following an ace or 2.  The number of Jokers included in the pack should
    be agreed before the start of the game; the Princeton branch of the family recommends no more than four.

    Each Joker played has the effect of moving play forward by one player (not hand): $X_{A2} X_{B2} J^k J^k
    X_{E2} X_{F2}$.  The order of play used to define ``following'' is the one after the card was played, so
    $X_B K_C J^k X_A$ not $X_B K_C J^k X_C$.

    For example, if A plays a 2 and someone\footnote{Anyone but B -- remember that you can't use a
      Joker to avoid picking up.} plays a Joker, C must pick up 2 cards or play a 2 herself.  Butting in on a
    card is permitted even if Jokers have been played on it, although the Jokers are not applied to the butter-in - the situation is as if they had not been played.  For
    example, if B plays an ace, A adds a Joker, and then D butts in on the ace, E must pick up 4 cards.

    N.b. Because Jokers skip players they may be used to adjust who's playing in 10s mode:
    $X_C X_E X_A J^k X_D X_F$.

    N.b. Because playing a Joker is not a turn, the Joker isn't counted when counting for a 6:
    $6_D X_C X_B X_C X_D$ but $6_D X_C J^k X_A X_B X_C$.

    N.b. If it's your turn and you play a Joker, play passes to the next player.  This means that it's
    impossible to go out by playing $J^k X$, although it \textit{is} possible to play a Joker to skip the
    preceeding player, say ``last card'', and go out: $X_A J^k$ [C says ``last card''] $X_C$.
\end{description}

\section{Butting in}
\label{buttingIn}

Uncle Baiting is always played with at least two packs of cards, which means that there are duplicates of all
cards (e.g. two Queens of Hearts).  When a card is played and you have an identical one in your hand, you may
always play it --- this is known as `Butting In'. If you do play an identical card (e.g. a 3 of Spades on another 3 of Spades), it is always considered a butt-in.  Following a butt-in, play proceeds as if you'd been the
original player of the card, and the butt-in does \emph{not} count as a separate play (an important point when
considering 6s and 9s).  If you have two identical cards in your hand you may butt-in on yourself, but you
must play the two cards separately rather than putting them down together.

In two cases, the butt-in augments rather than replaces the original card:
an Ace or a 2.  As soon as an Ace is played the next hand starts to pick up
four cards;  when the butted-in Ace is played the player after the butter-in
must pick up the remainder of the initial four cards, and also four cards for
the second Ace;  the butter-in is required to say ``Sorry'' as usual.  You
may choose to butt-in at any time before the player following
the original picker-upper plays.   Butting in with a 2 is similar;  the balance
of cards being picked up by the initial victim is transferred to the player
after the butter-in, along with an extra two for the butted-in 2.

When butting-in on a Jack, you may play either a Jack of the face-value or nominated suit. For
example, after a Jack of Hearts announced as ``Spades'' you may butt-in with either
a Jack of Hearts or a Jack of Spades.

Butting in in Queens mode is always considered to be a play from the trailing hand (i.e. you
don't get to play a second card). As previously mentioned, if a card can be a butt-in it is, so if, on their trailing hand, the person behind you played a 3 of spades, and you play another, you do not get to play a second card afterwards.

If C has gone out on an Ace or a 2, and B butts in, then it is counted as the same turn, and C is required to pick up the cards that would be normally required: $A_C A_B \Pi_C$ or $2_C 2_B \Pi_C$. Likewise, jokers still apply - for example: $A_C A_A J^k Pickup_C$ or $2_C 2_A Pickup_C$.

\section{Penalties}
\label{penalties}

Penalty cards are awarded for all errors; when something must be done promptly (e.g. saying ``Sorry'' or
``Last card'') this must be done before the next player plays to avoid a penalty.  Picking up a penalty
doesn't end your turn; after accepting it you must still play or pick up.

Examples of errors are:

\begin{itemize}
  \item Playing slowly;  the definition of `slowly' is left to the consensus of the other players.

  \item Attempting to play when it isn't your turn (e.g. if the player before you plays an 8).

  \item Playing an illegal card (e.g. not following suit-or-number;  playing a Jack on a red 7).

  \item Making a mistake even if the card was played illegally;  e.g. playing an Ace on a
    black 7 and forgetting to say ``Sorry'' would result in two penalty cards (plus potentially
    a third for arguing).

  \item Asking for clarification of the state of play;  after the card is awarded the
    clarification should be given.

  \item Explaining the state of play if the information has not been bought with a penalty.
    
  \item Dropping hints about cards that should be played.

  \item Being too officious about awarding penalty cards (the definition of `too' is
    to be decided by majority vote of the players).

  \item Failing to say ``Sorry'', ``In'', or ``Out'' as required when you triumphantly play your last card.  This is an
    especially satisfactory penalty, as it means that the game isn't over after all.
\end{itemize}

Once a penalty card has been awarded it may not be returned to the pickup-pile (attempting to do so will result in a penalty.)  If it is found that the award was
incorrect or unjust, the penalty is given to the person who originally proposed it.

\section{Uncle Baiting Junior}
\label{Junior}

As a gentle introduction for new Uncle Baiters, it's possible to play just a subset of the full rules:
\begin{itemize}
  \item Only Ace, 2, 7, 8, Jack, and King are special.

  \item All 7s are treated as being red (i.e. to be followed by 5 or less).
\end{itemize}

\section{Uncle Baiting Classic}
\label{Classic}

When Uncle Baiting came into the family it was a rather different game.

\begin{itemize}
  \item Only Ace, 2, 8, Jack, and King were special.

  \item Penalties were not enforced.

  \item There was no butting-in.
\end{itemize}

\section{Proposed Changes to the Rules}
\label{proposedChanges}

Uncle Baiting is an evolving game.  The following suggestions have been made, but
either not accepted, or we've been too scared to try them:

\begin{itemize}
  \item
    \begin{itemize}
      \item Split the players into two groups, initially the `even' and `odd' players counting round the table.
      \item Split the discard pile into two separate piles, one for each group.
      \item Each group plays a separate game of Uncle Baiting;  the first player to
        win either group is the overall winner.
      \item A player may butt-in to either group;  he or she then becomes a member of
        that group.
    \end{itemize}

    N.b. We haven't tried this one yet    

\end{itemize}

\end{document}
